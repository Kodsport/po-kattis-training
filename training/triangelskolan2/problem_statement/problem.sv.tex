\problemname{Triangelskolan 2}

Efter några år i Triangelskolan börjar Kajsa tycka det är tråkigt att bara göra en triangel i taget. Hon börjar fundera på om hon utifrån ett visst antal brickor kan skapa en {\em serie} av trianglar där sidlängderna bildar en obruten sekvens från $k$ till $m$, där $k\le m$. I denna uppgift måste hon använda alla brickorna.

Exempelvis, om Kajsa har 31 brickor, så kan hon göra triangelsekvensen 3, 4, 5, eftersom hon då använder $6+10+15=31$ brickor. Om hon däremot har $32$ brickor kan hon inte lösa problemet (1, 3, 4, 5 är inte godkänt eftersom det inte är en obruten sekvens).

\section*{Indata}
En rad med ett heltal $N$ ($1 \leq N \leq 10^9$), antalet brickor Kajsa har.

\section*{Utdata}
Om det går att skapa en sådan sekvens, skriv en rad med de två heltalen $k$ och $m$: sekvensens början och slut. Annars, skriv en rad med strängen \texttt{NEJ}. Om det finns flera möjliga sekvenser kan du skriva vilken som helst av dem.

\section*{Poängsättning}
Din lösning kommer att testas på en mängd testfallsgrupper.
För att få poäng för en grupp så måste du klara alla testfall i gruppen.

\noindent
\begin{tabular}{| l | l | p{12cm} |}
  \hline
  \textbf{Grupp} & \textbf{Poäng} & \textbf{Gränser} \\ \hline
  $1$    & $20$      & $N \leq 1000$ \\ \hline
  $2$    & $35$      & $N \leq 10^6$ \\ \hline
  $3$    & $45$      & Inga ytterligare begränsningar. \\ \hline
\end{tabular}
