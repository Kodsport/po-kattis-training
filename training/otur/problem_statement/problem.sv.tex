\problemname{Otur}
I boken \emph{Otur och 13 sätt att undvika det på} är varje sida numrerad med heltal, i ordningen $1, 2, 3, ...$ osv. För att undvika otur har författaren dock
hoppat över samtliga sidnummer som innehåller $13$ i sig. Detta betyder att sidnumrena t.ex. går $11, 12, 14, 15$, eller $128, 129, 140, 141$ och $1128$, $1129$, $1140$.

Tryckeriet som ska trycka boken blev enormt förvirrade av denna omständliga numrering - hur ska de veta vilket nummer som ska tryckas på varje sida?

Här kommer du in. Skriv ett program för att avgöra vilket sidnummer som ska tryckas på den $N$:te sidan i ordningen.

\section*{Indata}
Den första raden innehåller heltalet $N$ ($1 \leq N \leq 10^9$), vilken sida du ska avgöra sidnumret för.

\section*{Utdata}
Skriv ut ett heltal: sidnumret för den $N$:te sidan.

\section*{Poängsättning}
Din lösning kommer att testas på en mängd testfallsgrupper. För att få poäng
för en grupp så måste du klara alla testfall i gruppen.

\noindent
\begin{tabular}{| l | l | p{12cm} |}
  \hline
  \textbf{Grupp} & \textbf{Poäng} & \textbf{Gränser} \\ \hline
  $1$    & $20$      & $N \leq 1000$ \\ \hline
  $2$    & $80$      & Inga ytterligare begränsningar. \\ \hline
\end{tabular}

