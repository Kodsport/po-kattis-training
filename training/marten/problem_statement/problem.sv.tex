\problemname{Mårtens DFS}
Du får givet en enkel, sammanhängande graf $G$ med $N$ hörn. Varje hörn i $G$ har numrerats med tal mellan $0$ och $N-1$.

Avgör om en lista med heltal $L$ är en giltig djupet först-sökning. En giltig djupet först-sökning är en som kan genereras av följande procedur, anropad med något argument mellan $0$ och $N - 1$:

\begin{verbatim}
DFS(start):
  skriv ut start
  sett[start] = true
  för varje v : grannar(start):
    om inte sett[start]:
      DFS(v)
\end{verbatim}

Observera att grannarna till ett visst hörn kan gås igenom \emph{i vilken ordning som helst} av Mårtens algoritm.

\section*{Indata}
Den första raden innehåller två heltal $1 \le N \le 100\,000$ och $0 \le E \le 300\,000$, antalet hörn och kanter i $G$.

De nästa $E$ raderna innehåller två heltal $0 \le a, b \le N - 1$, som beskriver en kant i $G$ mellan hörnen $a$ och $b$.

Den sista raden kommer innehålla en (möjligtvis tom) blankstegs-separerad lista $L$ med heltal.

\section*{Utdata}
Du ska skriva ut \texttt{YES} om $L$ är en giltig djupet först-sökning. Annars ska du skriva ut \texttt{NO}.

\section*{Poängsättning}
Din lösning kommer att testas på en mängd testfallsgrupper. För att få poäng
för en grupp så måste du klara alla testfall i gruppen.

\begin{tabular}{| l | l | l | l |}
    \hline
    Grupp & Poängvärde & Gränser    \\ \hline
    1     & 19         & $1 \le n \le 8$ och $1 \le m \le 28$ \\ \hline
    2     & 43         & $1 \le n \le 1\,000$ och $1 \le m \le 5\,000$ \\ \hline
    3     & 38         & $1 \le n \le 100\,000$ och $1 \le m \le 300\,000$ \\ \hline
\end{tabular}

