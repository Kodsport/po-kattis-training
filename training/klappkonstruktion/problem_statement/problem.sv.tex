\problemname{Klappkonstruktion}
Såhär till jultider har Liselott Tomte mycket att göra. Granen ska kläs, julmat ska lagas, paket ska slås in och tomtens skägg ska ansas. Julnissarna springer runt som yra små höns och förlitar sig på Liselotts organisatoriska färdigheter för att något ska bli gjort.

I verkstaden finns det $N$ nissar, som tillsammans ska konstruera $M$ julklappar till världens barn. 
 Det finns tre sorters julklappar: små, medelstora och gigantiska. Konstruktionen av en liten klapp kräver bara att en enda nisse jobbar med klappen under en hel dag. Medelstora klappar behöver två nissar som jobbar med klappen, och båda nissar måste jobba under två hela dagar, och analogt så måste tre nissar jobba under tre dagar för att konstruera den gigantiska julklappen.

Liselott ska nu bestämma ett schema för att konstruera alla klappar så snart som möjligt. Vad är det minsta antalet dagar det kommer ta innan alla klappar är färdiga?

\section*{Indata}
Den första raden innehåller heltalet $N$ ($3 \leq N \leq 20$), antalet nissar i verkstaden.

På nästa rad finns tre tal $A$, $B$ och $C$. $A$ är antalet klappar som tar 1 dag att konstruera, $B$ antalet som kräver 2 dagar, och $C$ antalet som kräver 3 dagar. Det totala antalet klappar understiger alltid 50.

\section*{Utdata}
Skriv ut ett heltal: det minsta antalet dagar det tar innan samtliga klappar är färdigkonstruerade.


\section*{Poängsättning}
Din lösning kommer att testas på flera testfall. För att få 100 poäng så måste du klara alla testfall.
