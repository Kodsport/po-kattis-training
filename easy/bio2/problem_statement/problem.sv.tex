\problemname{Biosällskap 2}

Stockholms Förenade Biosällskapsförbund har planerat in en visning av \emph{Gamla dataloger och deras tårtningar} på den lokala KTH-bion.

Inte förrän det var försent påpekade förbundets revisor att styrelsen hade bokat in på tok för många biosällskap i salongen, som rymmer total $N$ biobesökare.

Det finns totalt $M$ biosällskap som anmält sig till visningen. Biosällskapen kommer att få komma in i salongen i ordningen de anmälde sig till visningen. Om det finns för få platser när ett sällskap försöker komma in så \emph{kommer biosalongen stänga och alla sällskap som är kvar kommer få gå hem.}

Givet sällskapens storlek, avgör hur många sällskap som inte blir insläppta.

\section*{Indata}
Den första raden i indatan innehåller heltalen $1 \le N \le 100$ och $1 \le M \le 50$, antalet platser i salongen respektive antalet sällskap.

Den andra raden innehåller $M$ heltal $1 \le S_i \le 10$. $S_i$ är storleken på det $i$:te sällskapet. Sällskapens storlekar är givna i anmälningsordning.


\section*{Utdata}
Utdatan ska bestå av ett enda tal, antalet sällskap som inte fick komma in på visningen.
