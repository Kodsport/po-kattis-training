\problemname{Växla pengar}
Martin Andræ i TEKVET16 spelar hellre pingis än lär sig räkna under matematiklektionerna. Och som
vi alla vet är missade mattelektioner inte något att sträva efter. Dessvärre har detta straffat Martin.

Han har nu fått sitt första sommarjobb i kassan på ICA här i Finspång. Ditt uppdrag är att hjälpa Martin
att ta reda på det minsta antalet mynt som ska växlas tillbaka. I kassaapparaten finns endast valörerna
$1$, $5$ och $20$.

Ditt uppdrag är nu att skriva en kod som hjälper Martin med att växla pengarna som blir över då kunden betalar med $100$
kr.

\section*{Indata}
Den första och enda raden av indata innehåller ett heltal $N$ ($1 \leq N < 100$), totalkostanden
för kundens varor, som ska betalas med $100$kr.

\section*{Utdata}
Skriv ut ett heltal: minsta antalet mynt som krävs för att ge kunden sin växel.


\section*{Poängsättning}
Din lösning kommer att testas på flera testfall. För att få 100 poäng så måste du klara alla testfall.


\section*{Förklaring av exempelfall 1}
I detta fallet ska $76$kr av varor betalas med $100$kr, vilket innebär att kunden ska få $24$kr växel.
Minsta antalet valörer är då en tjuga fyra enkronor.
