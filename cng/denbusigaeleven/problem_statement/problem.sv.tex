\problemname{Den busiga eleven}
En elev på Curt Nicolin Gymnasiet har precis kommit hem från en utlandsresa. Under
den tiden hade elevens klass räknat med exponenter. Eleven vill nu inte hamna efter
i sina studier jämfört med de övriga klasskamraterna och behöver därmed ett fusk som gör uträkningen åt honom.

Din uppgift är att skapa ett program som tar reda på värdet av $X$ upphöjt till $Y$ för $N$ stycken uppgifter.

\section*{Indata}
Den första raden innehåller ett heltal $N$ ($1 \leq N \leq 100$), antalet uppgifter.

De följande $N$ raderna innehåller vardera heltalen $X$ och $Y$ ($1 \leq a,b \leq 10$), 
vilket betyder att den här uppgiften är att beräkna $X^Y$.

\section*{Utdata}
För varje av de $N$ uppgifterna, skriv ut värdet av $X^Y$ på sin egen rad.

\section*{Poängsättning}
Din lösning kommer att testas på flera testfall. För att få 100 poäng så måste du klara alla testfall.

