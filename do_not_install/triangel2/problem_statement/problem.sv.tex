\problemname{Triangelskolan 2}

Efter några år i Triangelskolan börjar Kajsa tycka det är tråkigt att bara göra en triangel i taget. Hon börjar fundera på om hon utifrån ett visst antal brickor kan skapa en {\em serie} av trianglar där sidlängderna bildar en obruten sekvens från $k$ till $m$, där $k\le m$. I denna uppgift måste hon använda alla brickorna.

Exempelvis, om Kajsa har 31 brickor, så kan hon göra triangelsekvensen 3, 4, 5, eftersom hon då använder $6+10+15=31$ brickor. Om hon däremot har $32$ brickor kan hon inte lösa problemet (1, 3, 4, 5 är inte godkänt eftersom det inte är en obruten sekvens).

\section*{Indata}
En rad med ett heltal $N$, antalet brickor Kajsa har, där $1\le N \le 1\,000\,000\,000$.
\section*{Utdata}
Om det går att skapa en sådan sekvens, skriv en rad med de två heltalen $k$ och $m$: sekvensens början och slut. Annars, skriv en rad med strängen NEJ. Om det finns flera möjliga sekvenser kan du skriva vilken som helst av dem.

