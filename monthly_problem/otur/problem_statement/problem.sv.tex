\problemname{Otur}
I boken \emph{Otur och 13 sätt att undvika det på} är varje sida numrerad med heltal, i ordningen $1, 2, 3, ...$ osv. För att undvika otur har författaren dock
hoppat över samtliga sidnummer som innehåller $13$ i sig. Detta betyder att sidnumrena t.ex. går $11, 12, 14, 15$, eller $128, 129, 140, 141$ och $1128$, $1129$, $1140$.

Tryckeriet som ska trycka boken blev enormt förvirrade av denna omständliga numrering - hur ska de veta vilket nummer som ska tryckas på varje sida?

Här kommer du in. Skriv ett program för att avgöra vilket sidnummer som ska tryckas på den $N$:te sidan i ordningen.

\section*{Indata}
Den första raden innehåller ett enda heltal - $N \le 10^9$, vilken sida du ska avgöra sidnumret för.

\section*{Utdata}
Du ska skriva ut ett enda tal - sidnumret för den $N$:te sidan.
