\problemname{Långtbortistanska personnummer}

Du ska skriva ett program som räknar ut kontrollsiffran av ett långtbortistanskt personnummer.

Ett långtbortistanskt personnummer har 10 siffror. Den sista är en kontrollsiffra som räknas ut från de föregående siffrorna. Du ska ta emot de första 9 siffrorna, och ska skriva ut vilken kontrollsiffran är.

Kontrollsiffre-algoritmen defineras på följande vis:

\begin{verbatim}
Läs in 9 siffror i A[1...9]
Låt summa = 0
För varje siffra s = A[1], A[3], A[5], A[7], A[9]:
        Låt summa = summa + 3*s
För varje siffra s = A[2], A[4], A[6], A[8]:
        Låt summa = summa + 7*s
Låt kontrollsiffra = {resten när summa delas med 10}
Skriv ut kontrollsiffra
\end{verbatim}

\section*{Indata}
Indatan består av en rad; 9 stycken heltal mellan 0 och 9 separerade med blanksteg.

\section*{Utdata}
Skriv ut ett heltal mellan 0 och 9, personnummrets kontrollsiffra.

\section*{Poängsättning}
Din lösning kommer att testas på flera testfall. För att få 100 poäng så måste du klara alla testfall.
