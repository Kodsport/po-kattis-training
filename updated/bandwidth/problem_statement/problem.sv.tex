\problemname{Bandbredd}
Ett datornätverk består av $N$ datorer med $M$ olika direkta länkar mellan vissa par av datorer. Varje länk $i$ har en \emph{bandbredd} $b_i$.

Bandbredden för sekvens av länkar $a_1, a_2, ... a_k$ är minimum av bandbredden hos de individuella länkarna, dvs $min(b_{a_1}, b_{a_2}, ..., b_{a_k})$. En sekvens av länkar med bandbredder $4, 3, 1, 2$ har således bandbredd 1.

Givet två datorer $a$ och $b$ låter vi $B(a, b)$ vara den maximala bandbredden hos alla möjliga vägar av länkar mellan $a$ och $b$. Den \emph{minimala bandbredden} i hela nätverket är det minsta värdet som $B(a, b)$ antar över alla par av datorer. Så om $B(0, 1) = 1$, $B(1, 2) = 4, B(0, 2) = 3$ så är den minimala bandbredden 1.

\section*{Indata}
Den första raden innehåller två heltal $1 \le N \le 10^5$ och $0 \le M \le 3 \cdot 10^5$, antalet datorer och länkar i nätverket. 

De nästa $M$ raderna innehåller tre heltal $0 \le a, b \le N - 1$ och $1 \le c \le 3 \cdot 10^5$ som beskriver en direkt länk mellan datorerna $a$ och $b$ med bandbredd $c$.

Indatan är konstruerad så att det alltid finns en väg av länkar mellan vilket par av datorer som helst.

\section*{Utdata}
Du ska skriva ut ett heltal $B$ - den minimala bandbredden i nätverket.

\section*{Poängsättning}
Din lösning kommer att testas på en mängd testfallsgrupper. För att få poäng
för en grupp så måste du klara alla testfall i gruppen.

\noindent
\begin{tabular}{| l | l | p{12cm} |}
  \hline
  \textbf{Grupp} & \textbf{Poäng} & \textbf{Gränser} \\ \hline
  $1$    & $23$      & $2 \le n \le 8$, $1 \le m \le 28$ \\ \hline
  $2$    & $30$      & $2 \le n \le 1\,000$, $1 \le m \le 5\,000$ \\ \hline
  $3$    & $47$      & Inga ytterligare begränsningar. \\ \hline
\end{tabular}

